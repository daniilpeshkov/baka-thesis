\chapter{Описание файлов проекта}\label{appendix-extra-examples}%

Все исходные коды находятся в репозитории на Github \url{https://github.com/daniilpeshkov/calsoc}\\

\noindent \emph{./rtl/} -- директория с исходными кодами аппаратных описаний\\
%\noindent \emph{./rtl/gowin\_rpll/} -- подключение примитивов rPLL\\
%\noindent \emph{./rtl/measure\_unit/} -- модуль для проведения измерений\\
%\noindent \emph{./rtl/mem/} -- RAM и ROM\\
\noindent \emph{./firmware/} -- директория с исходными кодами управляющей программы\\
\noindent \emph{./bootloader/} -- директория с исходными кодами загрузчика\\
\noindent \emph{./syn/} -- директория с файлами для синтеза (назначения выводов ПЛИС, временные ограничения, конфигурации втроенного
логического анализатора)

\noindent \emph{./boot.py} -- скрипт для прошивки калибратора\\
\noindent \emph{./calsoc.gprj} -- файл для открытия проекта в Gowin EDA\\
