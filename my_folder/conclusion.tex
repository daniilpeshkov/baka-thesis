\chapter*{Заключение} \label{ch-conclusion}
\addcontentsline{toc}{chapter}{Заключение}	% в оглавление 

В результате работы получено управляющее устройство для калибратора субнаносекундной синхронизации.
Поставленные задачи: портирование под целевую ПЛИС и интеграция в СнК открытых модулей, создание
модулей для проведения измерений при помощи стробирования, написание и отладка встраиваемого программного обеспечения -- выполнены в полном объёме.\\

\noindent В процессе работы были освоены:
\begin{itemize}
	\item Маршрут проектирования на ПЛИС пр-ва Gowin Semiconductor
	\item Открытая шина для СнК Wishbone и разработка ведомых устройств для неё
	\item Применение открытого синтезируемого ядра на базе архитектуры RISC-V -- PicoRV32
	\item Проектирование и реализация СнК на основе открытых модулей
	\item Конфигурация компилятора и компоновщика под используемую ISA и СнК\\
\end{itemize}

На данный момент калибратор находится на этапе отладки, планируется вторая ревизия платы с исправлением ошибок, найденных в ходе отладки.
В планах на дальнейшую разработку имеется доработка прошивки, связанная с:
\begin{itemize}
	\item Улучшением метода определения частоты измеряемого сигнала и генерации строб
	\item Доработкой алгоритма и реализации снятия осциллограммы
	\item Добавлением интеграции в инфраструктуру White Rabbit\\
\end{itemize}

Также планируется проведение процедуры калибровки разработанного устройства, что позволит повысить точность проводимых измерений.
