\chapter{Разработка программно-аппаратной части управляющего устройства}

Все исходные коды аппаратных описаний находятся в директории \emph{/rtl/}.

\section{Описание верхнего уровня}

Описание верхнего уровня находится в файле \emph{/rtl/calsoc\_top.sv}.\\

Управляющее устройство реализовано в виде СнК (Система на кристалле) (\firef{fig:calsoc}) на базе открытого процессорного
ядра \emph{PicoRV32}, основанного на открытой архитектуре RISC-V.

Все периферийные модули подключаются к ядру через шину \emph{Wishbone}. Арбитраж на шине выполняет открытый модуль
\emph{wbxbar}.

\begin{figure}[ht!] 
	\center
	\includegraphics [scale=0.7] {my_folder/images//calsoc}
	\caption{Структурная схема СнК} 
	\label{fig:calsoc}  
\end{figure}

Все периферийные устройства разделяют между собой общее адресное пространство.

\begin{itemize}[label={}]
	\item 0x00000000 -- 0x00FFFFFF -- RAM 
	\item 0x01000000 -- 0x01FFFFFF -- ROM загрузичка
	\item 0x02000000 -- 0x02FFFFFF -- GPIO
	\item 0x03000000 -- 0x03FFFFFF -- UART1
	\item 0x04000000 -- 0x04FFFFFF -- память программы
	\item 0x05000000 -- 0x05FFFFFF -- измерительный модуль
\end{itemize}


\section{Измерительный модуль}

Измерительный модуль выдаёт все необходимые управляющие сигналы для проведения измерений и логически разделён на несколько
компонентов:

\begin{itemize}
	\item \emph{stb\_gen} -- модуль, измеряющий частоту сигнала и генерирующий стробы
	\item \emph{ch\_measure\_ctl} -- модуль, непосредственно управляющий измерением одного канала
	\item \emph{spi\_master} -- модуль, реализующий взаимодействие с AD5662
	\item \emph{sc\_fifo} -- FIFO, накапливающее измеренные значения
\end{itemize}

\begin{figure}[ht!] 
	\center
	\includegraphics [scale=0.7] {my_folder/images//mu_struct}
	\caption{Структурная схема измерительного модуля (\emph{measure\_unit})} 
	\label{fig:mu-struct}  
\end{figure}

Все описанные ранее модули подключены в модуле верхнего уровня \emph{measure\_unit} (\firef{fig:mu-struct}). В нём же
реализована вся логика управлением проведением измерений через шину Wishbone.

Далее будет отдельно рассмотрен каждый модуль.

\subsection{Модуль stb\_gen}

\begin{figure}[ht!] 
	\center
	\includegraphics [scale=0.3] {my_folder/images//stb_gen_big}
	\caption{Структурная схема блока генерации стробов (\emph{stb\_gen})} 
	\label{fig:stb-gen-big}  
\end{figure}


Для определения частоты измеряемого сигнала необходимо измерить время между двумя фронтами.
Для измерения используется 32-разрядный счётчик, один отсчёт которого равняется 8 нс (125 МГц).

Для измерения сигнала с периодом равным одной секунде необходима разрядность 27 бит,
однако для возможности измерять сигналы с меньшей частотой 
и кратности степени двойки разрядность увеличена до 32 бит.

Реализация <<в лоб>> не может работать стабильно на целевой ПЛИС из-за задержек на цепочке переносов в 
32-разрядном сумматоре. Для корректной работы на такой частоте необходимо конвейеризировать счётчик (\firef{fig:t-cnt}).

\begin{figure}[ht!] 
	\center
	\includegraphics [scale=0.7] {my_folder/images//t_cnt}
	\caption{Конвейерный 32-х разрядный счётчик} 
	\label{fig:t-cnt}  
\end{figure}

К младшим байтам (регистр \emph{LSB}) каждый такт прибавляется 1, текущее значение младших байт защёлкивается в регистре \emph{latched LSB}.
При переполнении, перенос защёлкивается в регистре \emph{latched carry}. К старшим байтам (регистр \emph{MSB}) прибавляется сохранённый перенос из-за
младших байт. Выход счётчика -- комбинация значений регистров \emph{MSB} и \emph{latched LSB}.


\lstset{
	numbersep = 5pt,
	stepnumber = 1
}


\begin{figure}[ht!] 
	\center
	\includegraphics [scale=0.7] {my_folder/images//stb_gen_fsm}
	\caption{Конечный автомат модуля \emph{stb\_gen}} 
	\label{fig:stb-gen-fsm}  
\end{figure}

\FloatBarrier

Нас \firef{fig:stb-gen-fsm} представлен конечный автомат модуля \emph{stb\_gen}. Вся логика максимально упрощена, так как, если
на одной регистровой передаче будет много комбинаторной логики, при синтезе под целевую ПЛИС не получится выдержать требуемые $ t_{setup} $ и $ t_{hold} $.

\begin{itemize}[label={}]
	\item FIND\_EDGE\_1 -- состояние после сброса, ожидание первого фронта на входе \emph{sig\_i} 
	\item FIND\_EDGE\_2 -- ожидание второго фронта на входе \emph{sig\_i} 
	\item WRITE\_START -- запись текущего значения \emph{t\_cnt} в \emph{t\_start}
	\item FIND\_EDGE\_3 -- ожидание третьего фронта на входе \emph{sig\_i}
	\item WRITE\_END -- запись текущего значения \emph{t\_cnt} в \emph{t\_end}
	\item COUNT\_PERIOD -- вычисление периода сигнала ($ \emph{t\_end} - \emph{t\_start}) $
	\item WAIT\_COUNT\_PERIOD -- задержка на 1 такт
	\item COUNT\_STROBE -- начало расчёта времени начала и конца отрицательного импульса на выходе строба
	\item WAIT\_STB\_END -- ожидание конца строба и переход к расчёту новой\\
\end{itemize}

При записи 1 в регистр управлением данным модулем происходит его сброс, и начинается определение частоты сигнала.

Помимо 32-х разрядного счётчика, конвейеризация была необходима для всех операций с 32-х битными числами. В модуле \emph{two\_cycle\_32\_adder}
реализован двухтактный сумматор, однако использованный подход идентичен тому, что используется в счётчике.

Отдельного упоминания стоит оптимизация операции проверки на равенство двух чисел. Для этого необходимо сравнить попарно все биты
чисел, что выполняется параллельно и не должно увеличивать максимальный путь, однако при синтезе получалась цепочка с последовательным
сравнением всех разрядов (\firef{fig:eq}).\\

\begin{figure}[ht!] 
	\center
	\includegraphics [scale=0.7] {my_folder/images//eq}
	\caption{Реализация сравнения при синтезе} 
	\label{fig:eq}  
\end{figure}

\FloatBarrier

Для исправления этого пришлось явно заменить операцию сравнения на исключающее ИЛИ и свёртку по ИЛИ-НЕ.

\lstset{
	numbersep = 5pt,
	stepnumber = 1
}

Выходом строба управляет арбитр, который держит компаратор в защёлкнутом состоянии. По приходе запроса на строб
(сигнал \emph{stb\_req\_i}) (\firef{fig:stb-gen-big}) арбитр пропускает один строб и выставляет логическую единицу на выходе
\emph{stb\_valid\_o}.





Для тестирования модуля были написаны тестбенчи \emph{tb\_stb\_gen.sv} и \emph{tb2\_stb\_gen.sv}. На \firef{fig:tb-stb-gen} и \firef{fig:tb-stb-gen-out}
представлены результаты моделирования.\\

\FloatBarrier

\begin{figure}[ht!] 
	\center
	\includegraphics  [scale=0.7] {my_folder/images//tb_stb_gen}
	\caption{Пример определения частоты входного сигнала} 
	\label{fig:tb-stb-gen}  
\end{figure}

\begin{figure}[ht!] 
	\center
	\includegraphics  [scale=0.8] {my_folder/images//tb_stb_gen_out}
	\caption{Вывод в консоль при симуляции} 
	\label{fig:tb-stb-gen-out}  
\end{figure}
\FloatBarrier

Видно, что сигналы с периодом, кратным 8 нс измеряются корректно, при измерении других -- ошибка меньше 8 нс.

\begin{figure}[ht!] 
	\center
	\includegraphics  [scale=0.3] {my_folder/images//stb_10mhz}
	\caption{Генерация стробов для 10 Мгц} 
	\label{fig:stb-10mhz}  
\end{figure}

\begin{figure}[ht!] 
	\center
	\includegraphics  [scale=0.3] {my_folder/images//stb_200khz}
	\caption{Генерация стробов для 200 кГц} 
	\label{fig:stb-200khz}  
\end{figure}

\FloatBarrier

На \firef{fig:stb-10mhz} и \firef{fig:stb-200khz} приведены осциллограммы генерации стробов для 10 МГц и 200 кГц.
Видно, что точно определить не кратную частоту невозможно -- при генерации получаются стробы с частотой 9.6 Мгц, а не 10 Мгц.

\subsection{Модуль \emph{ch\_measure\_ctl}}

Данный модуль управляет измерением одного из каналов -- выставляет необходимое пороговое напряжение и задержку, а
затем выставляет запрос на строб. После прохода строба, в зависимости от текущего и предыдущего выхода компаратора, принимается решение
о следующем значении задержки и порогового напряжения.

Для ускорения снятия осциллограммы применяется предсказание направления изменения сигнала. Для этого, после нахождения очередной точки,
изменяется только задержка, после чего на основании выхода компаратора принимается решение -- измеряемый сигнал нарастает или уменьшается.

Для запроса генерации строба используются сигналы \emph{stb\_req\_o} и \emph{stb\_valid\_i}, которые подключены к модулю \emph{stb\_gen}.

\begin{figure}[ht!] 
	\center
	\includegraphics  [scale=0.7] {my_folder/images//ch_ctl}
	\caption{Конечный автомат модуля \emph{ch\_measure\_ctl}} 
	\label{fig:ch-ctl-fsm}  
\end{figure}

\FloatBarrier

На \firef{fig:ch-ctl-fsm} представлен конечный автомат описываемого модуля.

\noindent
\begin{itemize}[label={}]
	\item IDLE -- состояние покоя 
	\item SET\_THRESHOLD -- отправление запроса на установку порогового напряжения ($ \emph{threshold\_wre\_o} = 1 $)
	\item WAIT\_THRESHOLD -- ожидание установки порогового напряжения ($ \emph{threshold\_rdy\_i} == 1 $)
	\item REQ\_STROBE -- установка запроса на стробу ($ \emph{stb\_req\_o} = 1 $)
	\item WAIT\_STROBE -- ожидание прохода строба ($ \emph{stb\_valid\_i} == 1 $)
	\item SAVE\_CMP\_RES -- обновление текущего и предыдущего выхода компаратора
	\item PROCESS\_RES -- принятие решения о статусе поиска точки (определение направления поиска или поиск точки)
					и о выборе направления поиска
	\item UPDATE\_CONF -- обновление регистров с текущей задержкой и пороговым напряжением на основании принятого ранее решения\\
\end{itemize}

В состоянии PROCESS\_RES, в зависимости от выхода компаратора переключаются два других конечных автомата, отвечающих за статус поиска точки и направления поиска.

На \firef{fig:p-find} приведён пример снятия осциллограммы сигнала.

\begin{figure}[ht!] 
	\center
	\includegraphics  {my_folder/images//p_find}
	\caption{Пример снятия осциллограммы с угадыванием направления изменения сигнала} 
	\label{fig:p-find}  
\end{figure}

\subsection{Регистры управления измерительным блоком}

Управление измерительным модулем осуществляется через регистры, размещённые в адресном пространстве процессора (memory-mapped) на шине Wishbone[14] ( \firef{fig:stb-reg}, \firef{fig:delta-reg},
\firef{fig:thr-reg}, \firef{fig:mu-ctl-reg}, \firef{fig:mu-p-reg}).

\begin{register}{H}{STB\_GEN\_CTL}{}% name=example
\label{example}%
\regfield{}{29}{3}{{reserved}}%
\regfield{clk\_sel}{1}{2}{{1/0}}%
\regfield{mux}{1}{1}{{1/0}}%
\regfield{run}{1}{0}{{1/0}}%
\reglabel{(w)} \regnewline%

\regfield{}{29}{3}{{reserved}}%
\regfield{clk\_sel}{1}{2}{{1/0}}%
\regfield{mux}{1}{1}{{1/0}}%
\regfield{rdy}{1}{0}{{1/0}}%
\reglabel{(r)}\regnewline%
\end{register}

\begin{register}{H}{STB\_GEN\_PERIOD}{}% name=example
\label{example}%
\regfield{period}{32}{0}{{read only}}%
\reglabel{(r)} \regnewline%
\end{register}

\begin{figure}[ht!] 
	\center
	\includegraphics  {my_folder/images//blank}
	\caption{Регистры для управления модулем \emph{stb\_gen}} 
	\label{fig:stb-reg}  
\end{figure}

%\noindent
%\begin{itemize}[label={}]
%	\item \textbf{run} -- запись 1 начинает измерение частоты входного сигнала
%	\item \textbf{mux} -- выбор канала для измерения (0 -- первый канал, 1 -- второй)
%	\item \textbf{clk\_sel} -- выбор источника тактового сигнала для генерации стробов (0 -- внутренний, 1 -- внешний)
%	\item \textbf{rdy} -- 1 при окончании измерения
%	\item \textbf{period} -- измеренное значение периода (кол-во отсчётов счётчика)\\
%\end{itemize}

\begin{description}
	\item [run] -- запись 1 начинает измерение частоты входного сигнала
	\item [mux] -- выбор канала для измерения (0 -- первый канал, 1 -- второй)
	\item [clk\_sel] -- выбор источника тактового сигнала для генерации стробов (0 -- внутренний, 1 -- внешний)
	\item [rdy] -- 1 при окончании измерения
	\item [period] -- измеренное значение периода (кол-во отсчётов счётчика)\\
\end{description}

\begin{register}{H}{CH\_CTL\_DELTA\_REG}{}% name=example
\label{example}%
\regfield{}{7}{26}{{reserved}}%
\regfield{delay delta}{10}{16}{{default: 0x01}}%
\regfield{threshold delta}{16}{0}{{default: 0x01}}%
\reglabel{(r/w)} \regnewline%

\end{register}


\begin{figure}[ht!] 
	\center
	\includegraphics  {my_folder/images//blank}
	\caption{Регистр для задания параметров измерения} 
	\label{fig:delta-reg}  
\end{figure}
\FloatBarrier

\noindent
\begin{itemize}[label={}]
	\item \textbf{threshold delta} -- шаг изменения порогового напряжения (разрешение по оси OY)
	\item \textbf{delay delta} -- шаг изменения задержки (разрешение по оси OX)\\
\end{itemize}


\begin{register}{H}{W\_THRESHOLD}{}% name=example
\label{example}%
\regfield{}{30}{2}{{reserved}}%
\regfield{dac2 rdy}{1}{1}{{1/0}}%
\regfield{dac1 rdy}{1}{0}{{1/0}}%
\reglabel{(r)} \regnewline%

\regfield{}{16}{16}{{reserved}}%
\regfield{threshold}{16}{0}{{}}%
\reglabel{(w)}\regnewline%
\end{register}

\begin{figure}[ht!] 
	\center
	\includegraphics  {my_folder/images//blank}
	\caption{Регистр для установки порогового напряжения} 
	\label{fig:thr-reg}  
\end{figure}
\FloatBarrier
\noindent
\begin{itemize}[label={}]
	\item \textbf{dac1/2 rdy} -- ЦАП на первом/втором канале установил указанное напряжение
	\item \textbf{threshold} -- запись в это поле устанавливает заданное напряжение на оба канала\\
\end{itemize}
\newpage
\FloatBarrier

\begin{register}{H}{MU\_CTL\_1/2}{}% name=example
\label{example}%
\regfield{}{31}{1}{{reserved}}%
\regfield{run}{1}{0}{{1/0}}%
\reglabel{(r/w)} \regnewline%
\end{register}

\begin{figure}[ht!] 
	\center
	\includegraphics  {my_folder/images//blank}
	\caption{Регистр для управление измерением канала} 
	\label{fig:mu-ctl-reg}  
\end{figure}
\FloatBarrier
\begin{itemize}[label={}]
	\item \textbf{run} -- запись 1 начинает измерение\\
\end{itemize}


\begin{register}{H}{MU\_CH\_1/2\_VAL}{}% name=example
\label{example}%
\regfield{}{15}{17}{{reserved}}%
\regfield{measured point}{16}{1}{{}}%
\regfield{valid}{1}{0}{{1/0}}%
\reglabel{(r)} \regnewline%
\end{register}

\begin{figure}[ht!] 
	\center
	\includegraphics  {my_folder/images//blank}
	\caption{Регистр для чтения измеренных значений} 
	\label{fig:mu-p-reg}  
\end{figure}
\FloatBarrier
\begin{itemize}[label={}]
	\item \textbf{measured point} -- значение измеренной точки
	\item \textbf{valid} -- 1 если FIFO не пустое и прочитано измеренное значение\\
\end{itemize}

\FloatBarrier

\section{Программная часть управляющего устройства}

Программная часть состоит из двух компонентов: загрузчика (\emph{/bootloader/}) и управляющей программы (\emph{/firmware/}).

Загрузчик необходим для обновления прошивки на FLASH памяти внутри калибратора. При запуске устройства загрузчик проверяет
включен ли переключатель на печатной плате калибратора, если нет, то исполнение передаётся управляющей программе.

Драйвера для используемой периферии находятся в директории (\emph{/firmware/src/dev/}). В заголовочном файле \emph{dev.h}
определены базовые адреса подключенной периферии.

Все измерения проводятся аппаратно, процессорное ядро используется для подачи команд на проведение измерений и
взаимодействия с устройством верхнего уровня, а так же отладки аппаратных модулей.
В дальнейшем планируется расширение программной части в связи с добавлением интеграции
White Rabbit.

Результат выводится через последовательный порт в консоль подключенного к калибратору ПК.

\FloatBarrier

\begin{figure}[ht!] 
	\center
	\includegraphics [scale=0.3] {my_folder/images//alg}
	\caption{Описание взаимодействия процессорного ядра и измерительного модуля} 
	\label{fig:alg}  
\end{figure}

\FloatBarrier

На \firef{fig:alg} изображена схема взаимодействия управляющей программы и измерительного модуля.
На начальном этапе запускается определение частоты, после этого подаётся команда на проведение
снятия осциллограммы. По готовности, измеренные точки считываются ядром.

\newpage