\chapter*{Введение} % * не проставляет номер
\addcontentsline{toc}{chapter}{Введение} % вносим в содержание

Для надёжного функционирования распределённых систем потоковой обработки данных, работающих в
режиме реального времени, на исполняемые в них операции накладываются строгие временные ограничения.
Многочисленные узлы таких систем могут находиться друг от друга на значительных расстояниях, 
что приводит к длительным (и не всегда одинаковым) задержкам при передаче сигналов между ними.
Возникающие задержки приводят к рассинхронизации работы устройств систем, что влечёт за собой 
возникновение потенциально некорректных результатов выполнения операций. 

С целью согласования всех устройств и обеспечения общего представления времени во всей сети применяются 
системы синхронизации часов. Такие системы гарантируют, что часы всех устройств сети отсчитывают время с 
одинаковой скоростью (выдают одинаковые показания в каждый момент времени).

Для передачи информации при синхронизации могут использоваться различные протоколы.
Наибольшее распространение получили следующие два: NTP (Network Time Protocol) и PTP (Precision Time Protocol).
Протокол NTP способен обеспечивать точность синхронизации времени до одной миллисекунды, а протокол PTP – до десяти миллисекунд. 

Протоколы NTP и PTP не подходят для случая, когда необходима синхронизация с субнаносекундной точностью. Например, 
такая высокая точность требуется в распределённых системах, применяемых в экспериментах физики высоких энергий.
Там они используются для потоковой обработки информации, поступающей с детекторов и ускорителей частиц. 

Системы субнаносекундной синхронизации применяются в Большом Адронном Коллайдере, расположенным в ЦЕРН, в Швейцарии. Они также
планируются к применению в строящемся комплексе «NICA» Объединённого института ядерных исследований (ОИЯИ) в Дубне. 
Другим приложением субнаносекундной синхронизации являются системы радиочастотного позиционирования,
использующие технологию сверхширокополосной связи (UWB) и алгоритм позиционирования TDoA (Time Difference of Arrival).

Основные положения работы описаны в [1].

\textbf{Цель работы:} разработка управляющего устройства для устройства, выполняющего калибровку систем субнаносекундной синхронизации.

\textbf{Решаемые в данной работе задачи:} портирование и интеграция существующих открытых модулей в систему на кристалле, 
реализация модулей для проведения измерений по принципу стробоскопического осциллографа, написание драйверов для используемой периферии,
написания управляющей программы для процессорного ядра, тестирование и отладка.

Разработка аппаратной части проекта выполнена на языке System Verilog, управляющей программы -- на языке C. 
Для компиляции используется компилятор gcc под архитектуру RV32IM.