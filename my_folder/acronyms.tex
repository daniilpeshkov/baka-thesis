\chapter*{Список сокращений и условных обозначений}             % Заголовок
%\addcontentsline{toc}{chapter}{Список сокращений и условных обозначений}  % Добавляем его в оглавление
\noindent
\addtocounter{table}{-1}% Нужно откатить на единицу счетчик номеров таблиц, так как следующая таблица сделана для удобства представления информации по ГОСТ
%\begin{longtabu} to \dimexpr \textwidth-5\tabcolsep {r X}
%\begin{longtabu} to \textwidth {r X} % Таблицу не прорисовываем!
% Жирное начертание для математических символов может иметь
% дополнительный смысл, поэтому они приводятся как в тексте
% диссертации
%\textbf{WoS} & Web of Science. \\
%\textbf{ВКР}  & Выпускная квалификационная работа. \\

%\end{longtabu}

\begin{tabular}{ccl}
WR & -- & White Rabbit\\
PTP & -- & Precision Time Protocol\\
УУ & -- & управляющее устройство\\
\end{tabular}