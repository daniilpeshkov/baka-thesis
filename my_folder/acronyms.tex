\chapter*{Список сокращений и условных обозначений}             % Заголовок
%\addcontentsline{toc}{chapter}{Список сокращений и условных обозначений}  % Добавляем его в оглавление
\noindent
\addtocounter{table}{-1}% Нужно откатить на единицу счетчик номеров таблиц, так как следующая таблица сделана для удобства представления информации по ГОСТ
%\begin{longtabu} to \dimexpr \textwidth-5\tabcolsep {r X}
%\begin{longtabu} to \textwidth {r X} % Таблицу не прорисовываем!
% Жирное начертание для математических символов может иметь
% дополнительный смысл, поэтому они приводятся как в тексте
% диссертации
%\textbf{WoS} & Web of Science. \\
%\textbf{ВКР}  & Выпускная квалификационная работа. \\

%\end{longtabu}
\noindent
\begin{tabular}{lcl}
КА & -- & Конечный автомат\\
ОУ & -- & Объект управления\\
ПЛИС & -- & Программируемая логическая интегральная схема\\
СнК & -- & Система на кристалле\\
УУ & -- & Управляющее устройство\\
ФАПЧ & -- & Фазовая автоподстройка частоты\\
BSRAM & -- & Burst Static Random Access Memory\\
DAC & -- & Digital to analog converter\\
FIFO & -- & first in, first out\\
GPIO & -- & General-purpose input/output\\
ISA & -- & Instruction Set Architecture\\
LUT & -- & Look-Up table\\
NTP & -- & Networking Time Protocol\\
PLL & -- & Phase-Locked Loop\\
PPS & -- & Pulse Per Second\\
PTP & -- & Precision Time Protocol\\
RAM & -- & Random Access Memory\\
ROM & -- & Read-only memory\\
SFP & -- & Small Form-factor Pluggable\\
UART & -- & Universal Asynchronous Receiver-Transmitter\\
WR & -- & White Rabbit\\
WR PTP & -- & White Rabbit Precision Time Protocol\\
\end{tabular}