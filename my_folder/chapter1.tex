\chapter{Анализ предметной области} \label{ch1}

\section{White Rabbit} \label{ch1:sec1}

White Rabbit -- система синхронизации часов. Разработана при сотрудничестве множества
институтов и компаний. Изначально проект был начат для улучшения текущей системы синхронизации в Церне.
Предполагалось использование для физических экспериментов, однако в процессе было создано обобщенное решение,
которое нашло своё применение в различных сферах.\\

\noindent Характеристики:

\begin{itemize}
	\item Суб-наносекундная точность
	\item Большое количество синхронизируемых узлов
	\item Расстояния в десятки километров
	\item Канал передачи между двумя узлами –- 1 Gbps
	\item Открытый исходный код\\
\end{itemize}

Достоинствами White Rabbit являются полностью открытый исходный код и аппаратура, а также 
использование существующих стандартов (Ethernet, PTP и т. д.).

\section{Калибровка}

Синхронизация в сети White Rabbit выполняется по протоколу WR PTP –- модифицированному протоколу PTP. (IEEE 1588).
Однако для достижения суб-наносекундной точности необходима дополнительная калибровка.

Обмен данными между двумя устройствами происходит по одной линии оптоволокна, работающей в полнодуплексном режиме.
Для передачи в одну и другую сторону используется свет с разной длиной волны, поэтому возникает асимметричность в
задержках распространения сигнала. Из-за этого устройство не может само определить задержки, отправив эхо запрос
другому устройству –- время распространения сигнала в одну и другую сторону не равны.

Определение коэффициента асимметричности оптоволокна позволит протоколу White Rabbit PTP обеспечить требуемую точность синхронизации устройств сети.

\begin{figure}[ht!] 
	\center
	\includegraphics  {my_folder/images//conn_model}
	\caption{Модель соединения между двумя устройствами} 
	\label{fig:conn-model}  
\end{figure}

На \firef{fig:conn-model} изображены возникающие задержки, требующие калибровки. Внутри каждого устройства 
возникают задержки на приёме и отправке ($\Delta_{TXM},\Delta_{RXM},\Delta_{TXS},\Delta_{RXS}, \varepsilon_{M},\varepsilon_{S}$),
которые являются результатом задержек в SFP (Small Form-factor Pluggable) 
модуле, в электрических цепях и электронных компонентах. Эти задержки калибруются отдельно на каждом устройстве и не являются
предметом рассмотрения в данной работе.

Суммарная задержка распространения сигнала от ведущего устройства к ведомому устройству и обратно считается по формуле:

\begin{equation}
delay_{MM} = \Delta_{TXM} + \Delta_{RXS} + \varepsilon_{S} + \Delta_{TXS} + \Delta_{RXM} + \varepsilon_{M} + \delta_{MS} + \delta_{SM}
\end{equation}

В данной работе рассматривается калибровка задержек $\delta_{MS}$ и $\delta_{SM}$ – задержек распространения сигнала по оптоволокну.

Когда оптоволокно ещё не установлено, измерить задержки и вычислить коэффициент асимметричности можно без особых усилий.
Коэффициент асимметричности определяется, как:

\begin{equation}
	\alpha = \frac{\delta_{MS} - \delta_{SM}}{\delta_{SM}}
\end{equation}

Измеряется коэффициент при помощи дополнительного короткого оптоволокна, с известной задержкой.

\begin{figure}[ht!] 
	\center
	\includegraphics [scale=0.4] {my_folder/images//meas_scheme_1}
	\caption{Измерение асимметричности про помощи дополнительного оптоволокна} 
	\label{fig:meas-scheme-1}  
\end{figure}

Два устройства подключаются калибруемым оптоволокном ($\delta_{2}$) и дополнительным ($\delta_{1}$) отдельно, 
далее два устройства синхронизируются. После этого измеряется разность фаз синхросигналов 1-PPS
(Pulse Per Second), генерируемых ведущим и ведомым устройством. 

\begin{equation}
	skew_{PPS} = t_{PPS_S} - t_{PPS_M}\\
\end{equation}

По измеренным значениям можно определить коэффициент асимметричности, как:

\begin{equation}
	\label{eq:alpha}
	\alpha = \frac{2 \left( skew_{PPS2} - skew_{PPS1} \right) }{\frac{1}{2} \delta_2 - \left( skew_{PPS2} - skew_{PPS1} \right)}
\end{equation}


Однако существуют так же уже установленные линии, нуждающиеся в калибровке. В таких случаях описанный выше метод не подходит.

\begin{figure}[ht!] 
	\center
	\includegraphics [scale=0.4] {my_folder/images//meas_scheme_2}
	\caption{Измерение асимметричности про помощи дополнительного оптоволокна} 
	\label{fig:meas-scheme-2}  
\end{figure}

В таких случаях применяется немного изменённый метод с подключением петли от выхода 1-PPS одного устройства к другому (\firef{fig:meas-scheme-2}). 
После синхронизации так же измеряются расхождения фронтов синхросигналов и усредняются.

\begin{equation}
	skew_{PPS} = \frac{1}{2} \left(skew_{PPS1} + skew_{PPS2} \right)
\end{equation}

Далее это значение может быть использовано для вычисления коэффициента по формуле \labelcref{eq:alpha}.

Разрабатываемое устройство служит для автоматизированного измерения разности фаз и выполнения калибровки.

\section{Актуальность}

Для процесса калибровки предполагается использовать устройство, способное измерять отрезки времени меньше, чем 1 нс, чтобы обеспечить
суб-наносекундную точность, т.е. иметь частоту дискретизации 1 GSa. Если применить правило <<пятикратного превышения частоты дискретизации>>, то
используемый осциллограф должен иметь частоту дискретизации выше 5 GSa. 

Цена осциллографов с такими характеристиками крайне высока. Актуальность данной работы заключается в том, что предлагается разработать
относительно бюджетное устройство, работающее по принципу стробоскопического осциллографа.
 
Сигналы PPS от калибруемых устройств являются периодическими. Это позволяет применить для их обнаружения, захвата и анализа устройство,
работающее по принципу стробоскопического осциллографа.

\section{Стробоскопический осциллограф}

Стробоскопические осциллографы предназначены для обнаружения, захвата и анализа периодических сигналов.
Принцип работы стробоскопического осциллографа проиллюстрирован на \firef{fig:stb-osc}.

\begin{figure}[ht!] 
	\center
	\includegraphics {my_folder/images//stb_osc}
	\caption{Измерение асимметричности про помощи дополнительного оптоволокна} 
	\label{fig:stb-osc}  
\end{figure}

Условные обозначения:
%$ U_{c} $ -- исследуемый периодический сигнал
\begin{itemize}[label={}]
	\item $ U_{c} $ -- исследуемый периодический сигнал
	\item $ T_{c} $ -- период исследуемого сигнала
	\item $ \tau $ -- длительность исмпульса исследуемого сигнала
	\item $ \Delta t $ -- шаг считывания исследуемого сигнала
	\item $ U_{2} $ -- стробы осциллографа
	\item $ U_{3} $ -- снятая осциллограмма сигнала\\
\end{itemize}

Таким образом, для снятия очередной точки изменяется смещение $ \Delta t $.
Частота дискретизации определяется минимальным смещением, которое может задать осциллограф.