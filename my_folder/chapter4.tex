\chapter{Разработка программной части управляющего устройства}

Программная часть состоит из двух компонентов: загрузчика (\emph{/bootloader/}) и управляющей программы (\emph{/firmware/}).

Загрузчик необходим для обновления прошивки на FLASH памяти внутри калибратора. При запуске устройства загрузчик проверяет
включен ли переключатель на печатной плате калибратора, если нет, то исполнение передаётся управляющей программе.

Драйвера для используемой периферии находятся в директории (\emph{/firmware/src/dev/}). В заголовочном файле \emph{dev.h}
определены базовые адреса подключенной периферии.

Все измерения проводятся аппаратно, процессорное ядро используется для подачи команд на проведение измерений и
взаимодействия с устройством верхнего уровня, а так же отладки аппаратных модулей.
В дальнейшем планируется расширение программной части в связи с добавлением интеграции
White Rabbit.

Результат выводится через последовательный порт в консоль подключенного к калибратору ПК.







\newpage